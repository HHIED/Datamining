%!TEX root=../main.tex

%%%%%%%%%%% GLOSSARY ENTRIES %%%%%%%%%%%
\newglossaryentry{NoSQL}
{
	name=\textit{No-SQL},
	plural=\textit{No-SQL},
	description={Not Only SQL: a database which relies on collections and documents rather than the classical relational databases with tables, columns and rows}
}

\newglossaryentry{Struct}
{
	name=\textit{Struct A/S},
	plural=\textit{Struct A/S},
	description={A company specializing in E-commerce solutions for businesses Struct has partnered with the project group and supplied the data for this Bachelor project}
}

\newglossaryentry{MongoDB}
{
	name=\textit{MongoDB},
	plural=\textit{MongoDB},
	description={A No-SQL database implementation}
}

\newglossaryentry{Amazon}
{
	name=\textit{Amazon},
	plural=\textit{Amazon},
	description={Amazon is a retail giant selling product to consumers in many different categories}
}

\newglossaryentry{EC2Instance}
{
	name=\textit{EC2 Instance},
	plural=\textit{EC2 Instances},
	description={A linux based virtual server running in the AWS cloud}
}

\newglossaryentry{ecommerce}
{
	name=\textit{e-commerce},
	plural=\textit{e-commerces},
	description={The business of buying and selling online}
}

\newglossaryentry{datamining}
{
	name=\textit{data mining},
	description={An analysis technique trying to discover useful information and relationships in large amounts of existing data}
}

\newglossaryentry{gls-API}
{
	name=\textit{API},
	description={An interface designed to integrate one piece of software with another}
}

\newglossaryentry{contentbased}
{
	name=\textit{Content-based filtering},
	description={A recommendation technique focusing on the content (description) of an item and how it relates to other items}
}

\newglossaryentry{knowledgebased}
{
	name=\textit{Knowledge-based recommendations},
	description={A recommendation technique relying on deep knowledge about the offered items}
}

\newglossaryentry{collaborativefiltering}
{
	name=\textit{Collaborative filtering},
	description={A recommendation technique linking items and users based on user-behavior, item descriptions etc.}
}

\newglossaryentry{hybrid}
{
	name=\textit{Hybrid recommendations},
	description={A mix of other recommendation techniques such is \gls{contentbased}, \gls{knowledgebased} and \gls{collaborativefiltering}}
}

\newglossaryentry{aspnet}
{
	name=\textit{ASP.NET Core},
	description={Open-source and cross-platform framework for building modern internet connected applications. It consists of modular components with minimal overhead to retain flexibility when constructing solutions}
}

\newglossaryentry{gls-REST}
{
	name=\textit{REST},
	description={An architectural style used for communication across systems on the internet}
}

\newglossaryentry{Docker}
{
	name=\textit{Docker},
	description={Docker is an open-source project that automates the deployment of applications inside software containers}
}

\newglossaryentry{amazonwebservice}
{
	name=\textit{Amazon Web-services},
	description={Provides on-demand cloud computing platforms}
}

\newglossaryentry{DockerContainer}
{
	name=\textit{Docker container},
	plural=\textit{Docker containers},
	description={A lightweight stand-alone executable package of a piece of software that includes everything needed to run it}
}




%%%%%%%%%%% ACRONYMS %%%%%%%%%%%

\newacronym[see={[Glossary:]{gls-API}}]{API}{\textit{API}}{Application Programming Interface\glsadd{gls-API}}

\newacronym[see={[Glossary:]{gls-REST}}]{REST}{\textit{REST}}{REpresentational State Transfer\glsadd{gls-REST}}

\newacronym{HTTP}{\textit{HTTP}}{HyperText Transfer Protocol}

\newacronym{XML}{\textit{XML}}{Extensible Markup Language}


