% Chapter Template

\chapter{Related work} % Main chapter title

\label{Chapter3} % Change X to a consecutive number; for referencing this chapter elsewhere, use \ref{ChapterX}

%----------------------------------------------------------------------------------------
%	SECTION 1
%----------------------------------------------------------------------------------------

\section{State of the art}

Datamining, webshop development and product recommendation algorithms are established parts of e-commerce development. This has resulted in great inspiration sources. In order to achieve the best possible result, some of the most successful developers of recommendation algorithms were researched. The video streaming service, Netflix, has invested a lot of resources in coming up with the best possible recommendation algorithm \cite{Netflix}. This includes a worldwide competition for \$1 million, called the Netflix Prize \cite{NetflixPrize}. Netflix can definitively be considered state of the art in the video streaming field. The retail giant, Amazon, is another company having great success with its product recommendation system. Amazon's product recommendation system plays a big part in increasing their sales \cite{AmazonSuccess}. What these two giants have in common, is that they are both using a collaborative recommendation algorithm. This is also the reason why the recommendation algorithm of this project is developed with the same technique, and will be elaborated further in chapter \ref{Chapter5}. Other techniques include content-based filtering, knowledge-based recommendations and hybrid recommendations. These techniques could be applied to the product, however, Content-based filtering is a good technique when recommending websites or articles. Knowledge-based recommendations is best put to use when high-involvement items are involved, such as cars, apartments, and financial services. Collaborative filtering best serves the purpose of product recommendations, due to its linking between users and items. Hybrid recommendations is a mix of all three methods \cite{recommendationtechniques}.


\section{Amazon}
Amazon has achieved great success with their recommendation system. There are many different techniques to develop a good product recommendation algorithm, but to develop one that is both smart, efficient and increases sale can be a difficult task. Some of the most common methods are user-to-user collaborative filtering, clustering and item-to-item collaborative filtering. Amazon's recommendation system is based on the latter, due to its fast pace response and precise recommendations. When developing a user-to-user collaborative filtering algorithm the result is often a precise, but slow recommendation system. By developing a clustering system, the response time can be very fast, but the quality of the recommendation will not be good \cite{AmazonRecommendations}. Other recommendation systems have been developed, but Amazon comes out as one of the greatest successors in the business and their recommendation system is one of their strong assets \cite{AmazonSuccess2}.
