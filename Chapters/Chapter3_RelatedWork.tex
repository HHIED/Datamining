% Chapter Template

\chapter{Related work} % Main chapter title

\label{RelatedWork} % Change X to a consecutive number; for referencing this chapter elsewhere, use \ref{ChapterX}

%----------------------------------------------------------------------------------------
%	SECTION 1
%----------------------------------------------------------------------------------------
This chapter covers different product recommendation approaches and the state of the art. \gls{Amazon} is introduced as they are a big player in the business of \gls{ecommerce}.

\section{State of the art}

\gls{datamining}, web shop development and product recommendation algorithms are established parts of \gls{ecommerce} development. This has resulted in great inspiration sources. In order to achieve the best possible result, some of the most successful developers of recommendation algorithms were researched. The video streaming service, Netflix, has invested a lot of resources coming up with the best possible recommendation algorithm \cite{Netflix}. This includes a worldwide competition for \$1 million, called the Netflix Prize \cite{NetflixPrize}. Netflix can definitively be considered state of the art in the video streaming field. The retail giant, \gls{Amazon}, is another company having great success with its product recommendation system. The product recommendation system developed by \gls{Amazon} plays a big part in increasing their sales \cite{AmazonSuccess}. What these two giants have in common, is that they are both using a collaborative recommendation algorithm. This is the reason why the recommendation algorithm of this project is developed with the same technique. Other techniques include content-based filtering, knowledge-based recommendations and hybrid recommendations. \gls{contentbased} is a good technique when recommending websites or articles. \gls{knowledgebased} is best put to use when it concerns high-involvement items, such as cars, apartments, and financial services. \gls{collaborativefiltering} best serves the purpose of product recommendations, due to its linking between users and items. Hybrid recommendations is a mix of all three methods \cite{recommendationtechniques}.


\section{Amazon}
\gls{Amazon} has achieved great success with their recommendation system. There are many different techniques to develop a good product recommendation algorithm, but to develop one that is both smart, efficient and increases sale can be a difficult task. Some of the most common \gls{collaborativefiltering} methods are user-to-user \gls{collaborativefiltering}, clustering and item-to-item \gls{collaborativefiltering}. Due to its fast pace response and precise recommendations, the recommendation system developed by \gls{Amazon} is based on the latter. When developing a user-to-user \gls{collaborativefiltering} algorithm the result is often a precise, but slow recommendation system. By developing a clustering system, the response time can be very fast, but the quality of the recommendation will not be good \cite{AmazonRecommendations}. Other recommendation systems have been developed, but \gls{Amazon} comes out as one of the greatest successors in the business and their recommendation system is one of their strong assets \cite{AmazonSuccess2}.
