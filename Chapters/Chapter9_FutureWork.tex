%!TEX root=../main.tex
% Chapter Template

\chapter{Future Work} % Main chapter title

\label{FutureWork} % Change X to a consecutive number; for referencing this chapter elsewhere, use \ref{ChapterX}

%----------------------------------------------------------------------------------------
%	SECTION 1
%----------------------------------------------------------------------------------------
This chapter highlights future areas of improvement to the system and how this would affect the solution.

\section{Future features}
This section focuses on the work which was not a priority in this version of the product. Some of these features have not been implemented due to the needs of the current customer, but could prove useful in the future. Other features have not been possible due to lacking data, such as feedback from actual users.

\subsection{Remaining requirements}
In the future the remaining requirements should be fulfilled. This will allow the companies using the \gls{API} to update and delete their data if, for example, a  product is no longer in their catalog. Deleting older behavior will keep the recommendations more up to date if the consumer patterns shift.

\subsection{Product ratings}
The data structure could be changed to accommodate web shops with the possibility of users rating the products. This will allow the algorithm to take the ratings into account to increase the accuracy of the product recommendations. Ratings on products allow more test measures to be calculated, see chapter \ref{Validation}.

\subsection{Order data}
In the future the algorithm could take already ordered items into account. With this data the algorithm can filter out products the visitor has already purchased. The \gls{collaborativefiltering} could also take orders into account. Adding orders to the similarity function would give a higher similarity score to products which have been ordered together rather than just looked at together. Order data can be combined with the visitor data.

\section{Feedback based improvement}
When the product is put to use and feedback starts generating from the users, the algorithm can use this feedback to improve the accuracy of the recommendations. If a user does not click on any of the recommendations the similarity function can be altered to try and remedy this. This can be further automated with the use of machine learning. A good machine learning implementation can make the algorithm learn from the generated feedback and automatically adjust some of the parameters to improve its own success rate \textit{"One  progressive  step  in  Recommender Systems (RS) history  is  the  adoption  of  machine  learning  (ML) algorithms, which allow computers to learn based on user information and to personalize recommendations  further."} \cite{RSAndML}. 

