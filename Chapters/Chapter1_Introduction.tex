% Chapter Template

\chapter{Introduction} % Main chapter title

\label{ChapterX} % Change X to a consecutive number; for referencing this chapter elsewhere, use \ref{ChapterX}

%----------------------------------------------------------------------------------------
%	SECTION 1
%----------------------------------------------------------------------------------------

This chapter covers the motivation behind the project and important areas relating to the project. It introduces the collaborating company who helped provide a case, data, and feedback. Finally the scope of the project is defined.

\section{Motivation}
The amount of data being processed on the server side and within large systems is continuously increasing. This data should be structured and modeled in a way that makes it easily accessible and easy to work with. Handling large amounts of data the right way can prove to be very useful, not only to the company who possesses the data, but also to the end users of a product. \Gls{datamining} is very useful in order to achieve this.
The company \gls{Struct} has provided a software engineering task of creating product recommendations where \gls{datamining} will create the foundation. This report will address the use of \gls{datamining}, the development of a solution that provides the user with intelligent product recommendations, and makes it possible to maintain current and future data.

\subsection{Data mining}
\gls{datamining} is an analysis technique trying to discover useful information and relationships in large amounts of existing data \cite{dataminingSource}. \\  
\gls{datamining} has become an important part of modern software engineering. Lots of companies tends to store large amount of data. If the data is analyzed properly and put to use, it can add tremendous value to the company as well as its users. In this case \gls{Struct} has stored information about users visiting one of their customers webshops. Previously, this data was stored in a database not optimized for product recommendations and not put to use. By processing the data properly, using \gls{datamining}, it can be structured in a way that makes it useful to the company e.g. product recommendations.

\color{black}
\subsection{Product recommendation}
If an \gls{ecommerce} company wants to increase its profit, product recommendation has proven to be very beneficial \cite{BigCommerce}. This is heavily used by multiple companies including the retail giant Amazon\cite{Fortune}. If you can predict what products your costumer may find useful, additional sales become more frequent. A data set like the one provided by \gls{Struct}, can make it possible to predict customer needs.

\section{Struct A/S}
\gls{Struct} is an IT-company specializing in developing \gls{ecommerce} solutions such as web shops and Product Information Management systems. The customers of \gls{Struct} are the web shop owners. \gls{Struct} has provided a data set from one of these customers containing information about the visitors of their web shop \cite{Struct}.


\section{Scope}
Product recommendation and \gls{datamining} are massive subjects consisting of much literature. Many large companies have tackled the challenge of providing recommendations for their users and spent a lot of time perfecting their algorithms.\\
This project will focus on designing and implementing a product recommendation algorithm capable of providing "good recommendations" The term "good recommendations" is a very loose term because it can vary from business to business or even from individual to individual. To directly compare the developed algorithm to others on the market would require access to these algorithms as well as a test on the same data. This has not been possible to acquire and therefore no direct comparison will be made. \\
In the limited time of this project it is not reasonable to expect \gls{Struct} to be able to begin using the API which means no online validation can be made. \\
The project comes with limited funds and as a result of this, some less optimal but free alternatives have been used for hosting the algorithm. \\
Only the requirements necessary to have a functional product are implemented as there is ample material to focus on.
