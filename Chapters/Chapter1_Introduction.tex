% Chapter Template

\chapter{Introduction} % Main chapter title

\label{ChapterX} % Change X to a consecutive number; for referencing this chapter elsewhere, use \ref{ChapterX}

%----------------------------------------------------------------------------------------
%	SECTION 1
%----------------------------------------------------------------------------------------

\section{Motivation}
The amount of data being processed around the Internet and within big systems is continuously increasing. This data should be structured and modelled in a way that makes it easily accessible and easy to work with. Handling large amounts of data the right way can prove to be very useful, not only to the company who possess the data, but also to the end users of a product. To achieve this, the art of data mining is very useful.
The company Struct A/S have provided a typical software engineering task where data mining will create the foundation. This report will address theoretical aspects about data mining, how it is done in practice and how the final results of the processed data can be put to use. \cite{MongoSQL}

\subsection{Data mining}
Data mining has become a big part of modern software engineering. Lots of companies tends to store large amount of data without structure and order within the data. This results in a lot of useless data which is both ineffective and a waste of resources. With prober data mining, it is possible to make this useless data useful to the company and its end users. In this case the data contain valuable information about users visiting the websites created and hosted by Struct A/S. By processing the data properly, it can be used for product recommendation, among other things.\color{red}(Find kilde på dette)

\color{black}
\subsection{Product recommendation}
If an e-commerce company wants to increase its profit, there is no doubt that product recommendation is one of the better ways of increasing your profit. This was first made popular by the retail giant Amazon. If you can predict what sorts of products your costumer may find useful, additional sales becomes more frequent. Big data sets, like the one provided by Struct A/S, can make it possible to predict customer needs, if the data is processed properly.\color{red}(Find kilde på dette) \color{black}