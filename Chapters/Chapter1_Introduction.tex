% Chapter Template

\chapter{Introduction} % Main chapter title

\label{ChapterX} % Change X to a consecutive number; for referencing this chapter elsewhere, use \ref{ChapterX}

%----------------------------------------------------------------------------------------
%	SECTION 1
%----------------------------------------------------------------------------------------

This chapter covers the motivation behind the project and some important areas relating to the project. It also introduces the collaborating company who helped provide a case and data. The chapter also defines the scope of the project.

\section{Motivation}
The amount of data being processed on the server side and within large systems is continuously increasing. This data should be structured and modeled in a way that makes it easily accessible and easy to work with. Handling large amounts of data the right way can prove to be very useful, not only to the company who possess the data, but also to the end users of a product. To achieve this, data mining is very useful.
The company Struct A/S has provided a software engineering task of creating product recommendations where data mining will create the foundation. This report will address the use of data mining, how to develop a solution that provides the user with intelligent product recommendations, and makes it possible to maintain current and future data.

\section{Struct A/S}
Struct A/S is an E-Commerce company specializing in developing E-commerce solutions such as web shops and Product Information Management systems. The customers of Struct A/S are the web shop owners. Struct A/S has provided us with a data set from one of these customers containing information about the visitors of their web shop \cite{Struct}.

\subsection{Data mining}
Data mining is an analysis technique trying to discover useful information and relationships in large amounts of existing data \cite{dataminingSource}. \\  
Data mining has become an important part of modern software engineering. Lots of companies tends to store large amount of data. If the data is analyzed properly and put into use, it can create tremendous value to the company as well as its users. In this case Struct has stored information about users visiting their websites. Previously, this data was stored in a database not optimized for product recommendations and not put to use. By processing the data properly, using data mining, it can be structured in a way that makes it useful to the company e.g. product recommendations.

\color{black}
\subsection{Product recommendation}
If an e-commerce company wants to increase its profit, product recommendation has proven to be very beneficial \cite{BigCommerce}. This is heavily used by multiple companies including the retail giant Amazon\cite{Fortune}. If you can predict what sorts of products your costumer may find useful, additional sales becomes more frequent. A data set like the one provided by Struct A/S, can make it possible to predict customer needs.

\section{Scope}
Product recommendation and data mining are massive subjects consisting of much literature. Many large companies have tackled the the challenge of providing recommendations for their users and spent a lot of time perfecting their algorithm.\\
This project will focus on designing and implementing a product recommendation algorithm capable of providing "good recommendation". The term "good recommendations" is a very loose term because it can vary from business to business or even from individual to individual. To directly compare the developed algorithm to others on the market would require access to these algorithms as well as a test on the same data. This has not been possible to acquire and therefore no direct comparison will be made. \\
In the limited time of this project it is not reasonable to expect Struct A/S to be able to begin using the API which means no online validation can be made. \\
The project comes with limited funds and as a result of this some less optimal but free alternatives have been used for hosting the algorithm. \\
Only the requirements necessary to have a functional product are to be implemented as there is ample material to focus on.