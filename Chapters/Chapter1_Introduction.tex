% Chapter Template

\chapter{Introduction} % Main chapter title

\label{ChapterX} % Change X to a consecutive number; for referencing this chapter elsewhere, use \ref{ChapterX}

%----------------------------------------------------------------------------------------
%	SECTION 1
%----------------------------------------------------------------------------------------

\section{Motivation}
The amount of data being processed around the Internet and within big systems is continuously increasing. This data should be structured and modeled in a way that makes it easily accessible and easy to work with. Handling large amounts of data the right way can prove to be very useful, not only to the company who possess the data, but also to the end users of a product. To achieve this, the art of data mining is very useful.
The company Struct A/S \cite{Struct} has provided us with a software engineering task of creating product recommendations where data mining will create the foundation. This report will address the use of data mining, how to develop a solution that provides the user with intelligent product recommendations, and makes it possible to maintain current and future data. \cite{MongoSQL}.

\subsection{Data mining}
Data mining is an analysis trying to discover useful information and relationships in large amounts of preexisting data \cite{dataminingSource}. \\  
Data mining has become a big part of modern software engineering. Lots of companies tends to store large amount of data. If the data is analyzed properly and put into use, it can create tremendous value to the company as well as its users. In this case Struct has stored information about users visiting their websites. Previously, this data was stored in a database not optimized for product recommendations and not put to use. By processing the data properly, using data mining, it can be structured in a way that makes it useful to the company e.g. product recommendations.

\color{black}
\subsection{Product recommendation}
If an e-commerce company wants to increase its profit, product recommendation has proven to be very beneficial\cite{BigCommerce}. This is heavily used by multiple companies including the retail giant Amazon\cite{Fortune}. If you can predict what sorts of products your costumer may find useful, additional sales becomes more frequent. Big data sets, like the one provided by Struct A/S, can make it possible to predict customer needs, if the data is processed properly.