% Chapter Template

\chapter{Problem statement} % Main chapter title

\label{Chapter1} % Change X to a consecutive number; for referencing this chapter elsewhere, use \ref{ChapterX}

%----------------------------------------------------------------------------------------
%	SECTION 1
%----------------------------------------------------------------------------------------

\section{Problem description}

The initial problem/challenge is given to us by the company Struct A/S and is described as follows: \\\\
When launching sites, whether it being regular websites or web shops, a lot of user activity is logged. We therefore have a large amount of data associated with each of our sites but do not currently use it. \\
In the future we would like to be able to use logged data to generate an insight into the user activity on our site and actively use this data to create a personalized experience for the users. \\\\

This project handles the initial analysis of the data, storing it in a scalable way and utilizing the data to create features which add value to the company. The focus of the project is data storing, data mining and recommendation algorithms. These methods are used to implement a final software solution capable of storing, organizing and utilizing current as well as new data about the end users. This allows Struct to easily keep their data updated and receive tailored product recommendations for their users.





\section{Problem statement}
The data we were given is in an unstructured format and can not be put to use as it is. This leads to the following problems - structuring and utilizing the data to create a personalized experience for the users, and making the data easily maintainable. \\
This leads to the following research questions:
\begin{itemize}
\item How can you optimally organize, store and access data in a scalable way?
\item How can this data be maintained and updated easily after deployment?
\item How can you utilize the organized data to generate tailored product recommendations for the end user?
\end{itemize}
