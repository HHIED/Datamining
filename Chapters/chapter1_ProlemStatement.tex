%!TEX root=../main.tex
% Chapter Template

\chapter{Problem} % Main chapter title

\label{Problem} % Change X to a consecutive number; for referencing this chapter elsewhere, use \ref{ChapterX}

%----------------------------------------------------------------------------------------
%	SECTION 1
%----------------------------------------------------------------------------------------
This chapter provides an overview of the problem description and states core research questions to be answered in this report.

\section{Problem description}

The initial case is provided by \gls{Struct} and is described as follows: \\

\textit{"When launching sites, whether it being regular websites or web shops, a lot of user activity is logged. We therefore have a large amount of data associated with each of our sites but do not currently use it.} \\
\textit{In the future we would like to be able to use logged data to generate an insight into the user activity on our site and actively use this data to create a personalized experience for the users."} \\

This project handles the initial analysis of the data, storing it in a scalable way and utilizing the data to create features which add value to the company. The focus of the project is data storing, \gls{datamining} and recommendation algorithms. These methods are used to implement a final software solution capable of storing, organizing and utilizing current as well as new data about the end users. This allows \gls{Struct} to easily keep their data updated and provide tailored product recommendations to the end-users.

\section{Problem statement}
The data was provided in a format not optimal for product recommendations, and can not be put to use as it is. This results in the following problems - structuring and utilizing the data to create a personalized experience for the users, and making the data easily maintainable. \\
The problems raise the following research questions:
\begin{itemize}
\item How can large amounts of data be optimally organized, stored and accessed in a scalable way?
\item How can this data be maintained and updated easily after deployment?
\item How can the organized data be utilized to generate tailored product recommendations for the end user?
\end{itemize}
