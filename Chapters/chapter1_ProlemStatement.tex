% Chapter Template

\chapter{Problem statement} % Main chapter title

\label{Chapter1} % Change X to a consecutive number; for referencing this chapter elsewhere, use \ref{ChapterX}

%----------------------------------------------------------------------------------------
%	SECTION 1
%----------------------------------------------------------------------------------------

\section{Problem description}

The initial problem/challenge  is given to us by the company Struct A/S and is described as follows: \\\\
When launching sites, whether it being regular websites or web shops, a lot of user activity is logged. We therefore have a large amount of data associated with each of our sites but do not currently use it. \\
In the future we would like to be able to use logged data to generate an insight into the user activity on our site and actively use this data to create a personalized experience for the users. \\\\

This project will handle the initial normalization of the data, storing it in a scalable way and utilizing the data to create features which add value for the company and the users. In order to achieve this, theory has to become implementation. Research is required in terms of data storing, data mining and recommendation algorithms. This research is implemented in the end system creating an API allowing Struct A/S to get useful information from the data such as the recommended products for a certain user. This API will be the final product and will utilize different technologies and algorithms.





\section{Problem statement}
The data we have been given is in a de-normalized format and the problem therefore comes with two challenges - normalizing the data and utilizing the data to create a personalized experience for the users. \\
This leads to the following research questions:
\begin{itemize}
\item How can you effectively normalize large amounts of data?
\item How can you optimally store and access data in a scalable way?
\item How can you utilize the data to generate useful features for the company and the end user?
\end{itemize}
