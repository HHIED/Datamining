% Chapter Template

\chapter{Discussion} % Main chapter title

\label{Discussion} % Change X to a consecutive number; for referencing this chapter elsewhere, use \ref{ChapterX}

%----------------------------------------------------------------------------------------
%	SECTION 1
%----------------------------------------------------------------------------------------
This chapter discusses key aspects of the implementation, how these affect the system and if other solutions were possible. The evaluation of the recommendations and the challenges related to this are also discussed.

\section{Data storage}
The first focus of the project was to organize a large datadump. This was done with the \gls{NoSQL} framework, \gls{MongoDB}. \gls{NoSQL} was chosen because of its great flexibility and its high performance even when the amount of data accumulates. With the amount of data used for \gls{datamining}, the project might also have been a success if a traditional SQL database had been used. However, \gls{Struct} asked for a scalable way of handling the data was it to exceed billions of records. In order to accommodate this requirement, \gls{NoSQL} was the better choice.

\section{Maintainability}
Another important issue was to create a system that allowed for easy maintenance. \gls{amazonwebservice} \gls{EC2Instance} served as the server for deploying the \gls{DockerContainer}. \gls{Docker} created an environment that will make later updates easy to apply or move the deployment to another server. The system itself has been developed in a way that allows each unique webshop to supply it with new data. New calculations will automatically be done when new data is stored. This ensures that the recommendation algorithm is always creating recommendations based on the newest information. Other web-services could have been used, but as a low-budget project \gls{Amazon} offered the best tools for free.
Instead of using \gls{Docker} the application itself could have been deployed on a windows server, and the system would be just as accessible. This would have made the deployment part of the project a lot easier, since we would not have to learn a new technology. Using the gls{DockerContainer} allows for cross-platform deployment, and will be easier to deploy elsewhere in the future if needed.

\section{Product recommendations}
The main objective was to develop a good quality product recommendation algorithm. The algorithm of the project did undergo a lot of changes during the process. A user-to-user \gls{collaborativefiltering} algorithm was first implemented, but resulted in slow response times. The clustering method was declined before implementation, because research indicated poor recommendations \cite{AmazonRecommendations}. In the end item-to-item \gls{collaborativefiltering} was applied, which meant more offline calculations. \\
The algorithm is meant to tailor the recommendations to each individual visitor. This is partly succeeded, but the item-to-item collaborative filter is based on the average customer behavior. This could result in poor recommendations for customers with more unique shopping habbits. Furthermore, if a new product is added, it would have to be visited by many different visitors before it would be recommended.

\section{Evaluating the algorithm}
One of the big challenges after implementing the recommendation algorithm was the evaluation. A comparison with \gls{Amazon}'s algorithm would be optimal, but \gls{Amazon} does not share their algorithm with the public. A comparison to other open-source algorithms could be done, but the amount of resources required to setup such an experiment would drift the focus of the project in a wrong direction. The recall and precision methods appeared as the best offline options. The recall rate was 53\% and the precision rate was 31\%. These measures indicate a somewhat successful recommendation system. The measures came with some issues as they do not show catalog coverage and the rates are penalized for recommendations not in the visitors behavior although these might have been good. Online validation of the recommendations would have been the best way to test if the recommendation algorithm is successful, however this was not possible in the limited scope. Finally a few concrete examples of product recommendations have been presented to illustrate the recommendations of the algorithm and these were in accordance with \gls{Struct} wishes. 