% Chapter Template

\chapter{Discussion} % Main chapter title

\label{Chapter7} % Change X to a consecutive number; for referencing this chapter elsewhere, use \ref{ChapterX}

%----------------------------------------------------------------------------------------
%	SECTION 1
%----------------------------------------------------------------------------------------
\section{Data storage}
The first focus of the project was to organize a lot of data. This was done with the No-SQL framework, MongoDB. No-SQL was chosen because of its great flexibility and its high performance even when the amount of data accumulates. With the amount of data used for datamining, the project might have been a success if a traditional SQL database had been used. However, Struct A/S asked for a scalable way of handling the data was it to exceed billions of records. In order to accommodate this requirement, No-SQL was the better choice.

\section{Maintainability}
Another important issue was to create a system that allowed for easy maintenance. Amazon web-services served as the tool for deploying the system, and Docker created an environment that will make later updates easy to apply. The system itself has been developed in a way that allows each individual client (unique webshop) to feed it with new data. New calculations will automatically be done when new data is stored. This ensures that the recommendation algorithm is always creating recommendations based on the newest information. Other web-services could have been used, but as a low-budget project Amazon offered the best tools for free.
Instead of using docker the application itself could have been deployed on a windows server, and the system would be just as easy accessible. This would have made the deployment part of the project a lot easier, since we would not have to learn a new technology. Using the Docker container allows for cross-platform deployment, and will be easier to deploy elsewhere in the future if needed.

\section{Product recommendations}
The main problem was to develop a good quality product recommendation algorithm. The algorithm of the project did undergo a lot of changing during the process. A user-to-user collaborative filtering algorithm was first implemented, but resulted in slow response times. The clustering method was declined before implementation, because research indicated that this would result in poor recommendations. \cite{AmazonRecommendations} In the end item-to-item collaborative filtering was applied, which meant more datamining and more offline calculations. If the initial research about recommendation technologies had been more thorough, valuable time would not have been wasted on implementing the wrong algorithm. However, the mistakes gave great educational value and the final algorithm was implemented in a short amount of time. Which was due to the understanding gained while developing the initial algorithm.\\
The algorithm is meant to tailor the recommendations to each individual visitor. This is partly succeeded, but the item-to-item collaborative filter is based on the average customer behavior. This could result in poor recommendations for customers with more unique shopping habbits. Furthermore, if a new product is added, it would have to be visited by many different visitors before it would be recommended.

\section{Evaluating the algorithm}
One of the big challenges after implementing the recommendation algorithm was the evaluation. A comparison with Amazon's algorithm would be optimal, but Amazon does not share their algorithm with the public. Even better would be an online evaluation, as mentioned in chapter \ref{Chapter6}. A comparison to other open-source algorithms could be done, but the amount of resources required to setup such an experiment would drift the focus of the project in a wrong direction. The recall method appeared as the best offline option, however did not give the best objective evaluation of the final system. Combining the concrete examples with the recall evaluation did, however, provide a more nuanced justification of the final product.